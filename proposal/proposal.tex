\PassOptionsToPackage{usenames}{color}
\documentclass[11pt]{article}

\usepackage{relsize} % relative font sizes (e.g. \smaller). must precede ACL style
\usepackage{acl2014}
\usepackage[linkcolor=blue]{hyperref}
\usepackage{natbib}
\newcommand{\citeposs}[1]{\citeauthor{#1}'s (\citeyear{#1})}

%\usepackage{times}
%\usepackage{latexsym}


\usepackage[boxed]{algorithm2e}
\renewcommand\AlCapFnt{\small}
\usepackage[small,bf,skip=5pt]{caption}
\usepackage{sidecap} % side captions
\usepackage{rotating}	% sideways

% Italicize subparagraph headings
\usepackage{titlesec}
\titleformat*{\subparagraph}{\itshape}
\titlespacing{\subparagraph}{%
  1em}{%              left margin
  0pt}{% space before (vertical)
  1em}%               space after (horizontal)

% Numbered Examples and lists
\usepackage{lingmacros}

\usepackage{enumitem} % customizable lists
\setitemize{noitemsep,topsep=0em,leftmargin=*}
\setenumerate{noitemsep,leftmargin=0em,itemindent=13pt,topsep=0em}


\usepackage{textcomp}
% \usepackage{arabtex} % must go after xparse, if xparse is used!
%\usepackage{utf8}
% \setcode{utf8} % use UTF-8 Arabic
% \newcommand{\Ar}[1]{\RL{\novocalize #1}} % Arabic text

\usepackage{listings}

\lstset{
  basicstyle=\itshape,
  xleftmargin=3em,
  aboveskip=0pt,
  belowskip=-3pt, %-.5\baselineskip, % correct for extra paragraph break inserted after listing
  literate={->}{$\rightarrow$}{2}
           {α}{$\alpha$}{1}
           {δ}{$\delta$}{1}
           {(}{$($}{1}
           {)}{$)$}{1}
           {[}{$[$}{1}
           {]}{$]$}{1}
           {|}{$|$}{1}
           {+}{\ensuremath{^+}}{1}
           {*}{\ensuremath{^*}}{1}
}

\usepackage{amssymb}	%amsfonts,eucal,amsbsy,amsthm,amsopn
\usepackage{amsmath}

\usepackage{mathptmx}	% txfonts
\usepackage[scaled=.8]{beramono}
\usepackage[T1]{fontenc}
\usepackage[utf8x]{inputenc}

\usepackage{MnSymbol}	% must be after mathptmx

\usepackage{latexsym}





% Tables
\usepackage{array}
\usepackage{multirow}
\usepackage{booktabs} % pretty tables
\usepackage{multicol}
\usepackage{footnote}


\usepackage{url}
\usepackage[usenames]{color}
\usepackage{xcolor}

% colored frame box
\newcommand{\cfbox}[2]{%
    \colorlet{currentcolor}{.}%
    {\color{#1}%
    \fbox{\color{currentcolor}#2}}%
}

\usepackage[normalem]{ulem} % \uline
\usepackage{colortbl}
\usepackage{graphicx}
\usepackage{subcaption}
%\usepackage{tikz-dependency}
%\usepackage{tikz}
%\usepackage{tree-dvips}
%\usetikzlibrary{arrows,positioning,calc} 
\usepackage{xytree}

\usepackage{xspace} % \xspace command for macros (inserts a space unless followed by punctuation)

\DeclareMathOperator*{\argmax}{arg\,max}
\DeclareMathOperator*{\argmin}{arg\,min}
\setlength\titlebox{6.5cm}    % Expanding the titlebox



% Author comments
\usepackage{color}
\newcommand\bmmax{0} % magic to avoid 'too many math alphabets' error
\usepackage{bm}
\definecolor{orange}{rgb}{1,0.5,0}
\definecolor{mdgreen}{rgb}{0,0.6,0}
\definecolor{mdblue}{rgb}{0,0,0.7}
\definecolor{dkblue}{rgb}{0,0,0.5}
\definecolor{dkgray}{rgb}{0.3,0.3,0.3}
\definecolor{slate}{rgb}{0.25,0.25,0.4}
\definecolor{gray}{rgb}{0.5,0.5,0.5}
\definecolor{ltgray}{rgb}{0.7,0.7,0.7}
\definecolor{purple}{rgb}{0.7,0,1.0}
\definecolor{lavender}{rgb}{0.65,0.55,1.0}

% Settings for algorithm listings
% \lstset{
%   language=Python,
%   upquote=true,
%   showstringspaces=false,
%   formfeed=\newpage,
%   tabsize=1,
%   commentstyle=\itshape\color{lavender},
%   basicstyle=\small\smaller\ttfamily,
%   morekeywords={lambda},
%   emph={upward,downward,tc},
%   emphstyle=\underbar,
%   aboveskip=0cm,
%   belowskip=-.5cm
% }
%\renewcommand{\lstlistingname}{Algorithm}


\newcommand{\ensuretext}[1]{#1}
\newcommand{\cjdmarker}{\ensuretext{\textcolor{green}{\ensuremath{^{\textsc{CJ}}_{\textsc{D}}}}}}
\newcommand{\nssmarker}{\ensuretext{\textcolor{magenta}{\ensuremath{^{\textsc{NS}}_{\textsc{S}}}}}}
\newcommand{\nasmarker}{\ensuretext{\textcolor{red}{\ensuremath{^{\textsc{NA}}_{\textsc{S}}}}}}
\newcommand{\lkmarker}{\ensuretext{\textcolor{blue}{\ensuremath{^{\textsc{L}}_{\textsc{K}}}}}}
\newcommand{\swswmarker}{\ensuretext{\textcolor{orange}{\ensuremath{^{\textsc{S}}_{\textsc{S}}}}}}
\newcommand{\abmarker}{\ensuretext{\textcolor{purple}{\ensuremath{^{\textsc{A}}_{\textsc{B}}}}}}
\newcommand{\arkcomment}[3]{\ensuretext{\textcolor{#3}{[#1 #2]}}}
%\newcommand{\arkcomment}[3]{}
\newcommand{\nss}[1]{\arkcomment{\nssmarker}{#1}{magenta}}
\newcommand{\aj}[1]{\arkcomment{\cjdmarker}{#1}{green}}
\newcommand{\dirk}[1]{\arkcomment{\nasmarker}{#1}{red}}
\newcommand{\lk}[1]{\arkcomment{\lkmarker}{#1}{blue}}
\newcommand{\swsw}[1]{\arkcomment{\swswmarker}{#1}{orange}}
\newcommand{\ab}[1]{\arkcomment{\abmarker}{#1}{purple}}
\newcommand{\wts}{\mathbf{w}}
\newcommand{\g}{\mathbf{g}}
\newcommand{\f}{\mathbf{f}}
\newcommand{\x}{\mathbf{x}}
\newcommand{\y}{\mathbf{y}}
\newcommand{\overbar}[1]{\mkern 1.5mu\overline{\mkern-1.5mu#1\mkern-1.5mu}\mkern 1.5mu} % \bar is too narrow in math
\newcommand{\cost}{c}

\newcommand{\Sref}[1]{\S\ref{#1}}
\newcommand{\fref}[1]{figure~\ref{#1}}
\newcommand{\ffref}[2]{figures~\ref{#1} and~\ref{#2}}
\newcommand{\Fref}[1]{Figure~\ref{#1}}
\newcommand{\FFref}[2]{Figures~\ref{#1} and~\ref{#2}}
\newcommand{\tref}[1]{table~\ref{#1}}
\newcommand{\ttref}[2]{tables~\ref{#1} and~\ref{#2}}
\newcommand{\Tref}[1]{Table~\ref{#1}}
\newcommand{\aref}[1]{algorithm~\ref{#1}}
\newcommand{\Aref}[1]{Algorithm~\ref{#1}}
\newcommand{\fnref}[1]{footnote~\ref{#1}}
\newcommand{\eref}[1]{eq.~\eqref{#1}}
\newcommand{\Eref}[1]{Eq.~\eqref{#1}}
\newcommand{\exref}[1]{(\ref{#1})} % numbered example

% Space savers
% From http://www.eng.cam.ac.uk/help/tpl/textprocessing/squeeze.html
% \addtolength{\dbltextfloatsep}{-.5cm} % space between last top float or first bottom float and the text.
% \addtolength{\intextsep}{-.5cm} % space left on top and bottom of an in-text float.
% \addtolength{\abovedisplayskip}{-.5cm} % space before maths
% \addtolength{\belowdisplayskip}{-.5cm} % space after maths
% %\addtolength{\topsep}{-.5cm} %space between first item and preceding paragraph
% \setlength{\belowcaptionskip}{-.25cm}


% customize \paragraph spacing
% \makeatletter
% \renewcommand{\paragraph}{%
%   \@startsection{paragraph}{4}%
%   {\z@}{.2ex \@plus 1ex \@minus .2ex}{-1em}%
%   {\normalfont\normalsize\bfseries}%
% }
% \makeatother


% Special macros
\newcommand{\tg}[1]{\texttt{#1}}	% supersense tag name
\newcommand{\gfl}[1]{%\renewcommand\texttildelow{{\lower.74ex\hbox{\texttt{\char`\~}}}} % http://latex.knobs-dials.com/
\mbox{\textsmaller{\texttt{#1}}}}	% supersense tag symbol
\newcommand{\lex}[1]{\textsmaller{\textsf{\textcolor{slate}{\textbf{#1}}}}}	% example lexical item 
\newcommand{\tagdef}[1]{#1\hfill} % tag definition
\newcommand{\tagt}[2]{\ensuremath{\underset{\textrm{\textlarger{\tg{#2}}}\strut}{\w{#1}\rule[-.3\baselineskip]{0pt}{0pt}}}} % tag text (a word or phrase) with an SST. (second arg is the tag)
\newcommand{\glosst}[2]{\ensuremath{\underset{\textrm{#2}}{\textrm{#1}}}} % gloss text (a word or phrase) (second arg is the gloss)
\newcommand{\AnnA}[0]{\mbox{\textbf{Ann-A}}} % annotator A
\newcommand{\AnnB}[0]{\mbox{\textbf{Ann-B}}} % annotator B
\newcommand{\sys}[1]{\mbox{\textbf{#1}}}   % name of a system (one of our experimental conditions)
\newcommand{\dataset}[1]{\mbox{\textsc{#1}}}	% one of the datasets in our experiments
\newcommand{\datasplit}[1]{\mbox{\textbf{#1}}}	% portion one of the datasets in our experiments

\newcommand{\w}[1]{\textit{#1}}	% word
\newcommand{\tweet}[1]{\textsf{#1}}	% tweet
\newcommand{\twbank}[0]{\textsc{Tweebank}\xspace}
\newcommand{\foster}[0]{\textsc{Foster}\xspace}
\newcommand{\twparser}[0]{\textsc{Tweeboparser}\xspace}
\newcommand{\tat}[0]{\textasciitilde}

%\newcommand{\finalversion}[1]{#1}
\newcommand{\finalversion}[1]{}
\newcommand{\shortversion}[1]{#1}
\newcommand{\considercutting}[1]{#1}
\newcommand{\longversion}[1]{} % ...if only there were more space...

\hyphenation{WordNet}
\hyphenation{WordNets}
\hyphenation{VerbNet}
\hyphenation{FrameNet}
\hyphenation{SemCor}
\hyphenation{PennConverter}
\hyphenation{TurboParser}
\hyphenation{Tweebo-parser}
\hyphenation{Twee-bank}
\hyphenation{an-aly-sis}
\hyphenation{an-aly-ses}
\hyphenation{news-text}
\hyphenation{base-line}
\hyphenation{de-ve-lop-ed}
\hyphenation{comb-over}

\title{Joint Tagging of Multiword Expressions and Supersenses}

% Lingpeng Kong, Nathan Schneider, Swabha Swayamdipta, Archna Bhatia, Chris Dyer, and Noah A. Smith 
\author{
Nathan Schneider \\
		School of Informatics\\
	   	University of Edinburgh\\
	    Edinburgh, EH8 9AB, UK\\
	    {\tt emailtba@inf.ed.ac.uk} \And
Dirk Hovy \quad Anders Johannsen\\
Center for Language Technology\\
University of Cophenhagen\\
Njalsgade 140, 2300 Copenhagen, Denmark\\
{\tt dirk@cst.dk}, {\tt ajohannsen@hum.ku.dk}}


\date{}

\begin{document}
\maketitle
\begin{abstract}
We propose that the CoNLL~2015 Shared Task 
consist of analyzing the lexical semantics of sentences 
in a broad-coverage fashion. 
The task formulation consists of two steps which can be performed jointly:
(a)~chunking tokens within sentences into \textbf{multiword expressions} (MWEs), and 
(b)~assigning coarse \textbf{supersense} labels to all noun and verb expressions. 
The evaluation will build upon existing annotated datasets in two social web genres.
We expect that the task formulation will foster engagement across 
multiple subcommunities of computational semantics, 
and facilitate empirical comparison of methodologically diverse systems.

\nss{Note: this document uses natbib rather than the standard ACL bib macros: 
According to \citet{baldwin-10}, statistical association measures 
are one technique for extracting MWE types \citep[cf.][\emph{inter alia}]{pecina-10}.}
\end{abstract}

\section{Introduction}

\nss{motivation: lexical semantic analysis of MWEs and supersenses in English. 
semantics is a hot topic.
highlights: (a) broad-coverage---not too far from NER; hopefully avoid some of the limitations of traditional WSD. 
(b) robust to domain---in fact, evaluation consists of 2 social web domains.
(c) potential to benefit from many different NLP techniques, including sequence tagging/chunking, 
parsing, distributional word representations, language models, 
use of language resources such as WordNet, (other buzzwords?)}

\nss{foster cross-pollination between different subcommunities}

\nss{encourage creativity of solutions}

\nss{related work: these representations build on ones that have been explored, 
but go a step beyond. (want to convince reader that there will be interest in the task based on previous results, 
but also that it is somewhat novel.)}

\nss{the proposed task will be limited to English. 
this reflects what is realistic given existing resources and the organizers' ability to annotate new data. 
if the task is successful, we would advocate a second, multilingual incarnation 
(note other languages that have seen MWE/supersense evaluations---Italian, Arabic, etc.)}

\section{Representation}

\nss{given a sentence, what should be predicted. extra examples in appendices are encouraged}

\nss{file format with example. UTF-8}

\section{Data}

\nss{synopsis of annotated datasets. how they were sampled. splits. what new (or revised) annotations are needed? 
(CPH should have resources to annotate a new Twitter test set with supersenses. possibly MWEs as well)
(Nathan's SST data is unreleased, but same sentences as CMWE corpus. 
so participants should be banned from using the CMWE data lest they train on the test set.)
(is it OK if, e.g., supersense tagging conventions are different in different datasets?)
be as specific as possible about timeline}

\nss{issue to resolve: licensing of Web Treebank source text}

\section{Evaluation}

\subsection{System Submission Process}

On May~5,~2015, participants will be furnished with the test data (minus the gold labels). 
They will have until May~10 to submit up to 3~system predictions for evaluation. 
The test data will include sentences from both evaluation domains, 
in a random order: to encourage robust systems, 
the domain of each sentence will not be marked at test time, 
and the proportion of sentences from each domain is not guaranteed to be 
the same in the trial, train, dev, and test sets.\nss{is this crazy?}

\subsection{Inputs}

\nss{open, closed track. if a team submits systems to the closed track and fewer than 3 systems to the open track, 
they will be encouraged to submit the closed track results to the open track as well.}

\nss{systems must respect the input tokenization}

\nss{domain label at training time, but not test time. no other metadata}

\nss{we'll allow WordNet and provide (auto?) POS, Brown clusters for closed track?}

\nss{semi-open track: we provide a large unlabeled corpus, in addition to closed track resources?}

\nss{unsupervised track (no token-level MWE or SST labels may be used)??}

\subsection{Measures}

\nss{several possibilities---I think we should use the bolded ones:
\begin{enumerate}
\item \textbf{MWE: link-based P, R, $F_1$ with strength averaging}
\item \textbf{SST first-tag supersense accuracy}
\item full tagging accuracy (with, without O)
\item MWE+SST chance-corrected: Kripp's unitizing $\alpha$ with MWE ID as one of the ``categories'', and chance levels determined by the gold annotations
\item \textbf{joint MWE+SST: link-based F1 with matching links count as 1 if the supersense agrees, .5 if it agrees in POS (noun/verb/aux/other), and .25 otherwise}
\end{enumerate}
}

\subsection{Conditions}

All measures will be reported separately for each of the two evaluation domains, 
and in a combined figure with the two domains weighted equally. 

Competitive rankings will be determined from the joint MWE+SST measure in the combination of the domains. 
There will be two such rankings: one in the closed track, and one in the open track.
\nss{At the discretion of the organizers, honorable mention awards may go to systems that place well on 
any of the measures, or that were especially creative in their approach?}

\subsection{Baselines}

\nss{we already have systems to serve as baselines, each customized to one domain}

\section{Conclusion}

\nss{TODO}

\bibliographystyle{aclnat}
% you bib file should really go here
\setlength{\bibsep}{10pt}
{\fontsize{10}{12.25}\selectfont
\bibliography{proposal}}



\end{document}
